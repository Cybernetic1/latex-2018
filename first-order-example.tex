\documentclass[12pt, orivec]{article}
\usepackage{amsmath}
\usepackage{wasysym}

\ifdefined\chinchin
\usepackage{xeCJK}
\setCJKmainfont[BoldFont=SimHei,ItalicFont=AR PL KaitiM GB]{SimSun}
\newcommand{\cc}[2]{#1}
\else
\newcommand{\cc}[2]{#2}
\fi

\begin{document}
\setlength{\parindent}{0pt}
\setlength{\parskip}{2.8ex plus0.8ex minus0.8ex}

例子:「如果 a 爱 b,而且 b 爱 a,则 a 开心」

这是 first-order formula:
\begin{equation}
a \heartsuit b \wedge b \heartsuit a \rightarrow \smiley a	
\end{equation}
改成 prefix notation:
\begin{equation}
\heartsuit a b \wedge \heartsuit b a \rightarrow \smiley a	
\end{equation}
应用以下的 combinators:
\begin{eqnarray}
& \mathsf{K} x y & \rightarrow x \\
& \mathsf{C} x y z & \rightarrow x z y
\end{eqnarray}
改写成这样:
\begin{equation}
\heartsuit a b \wedge \mathsf{C} \heartsuit a b \rightarrow \smiley \mathsf{K} a b
\end{equation}
再可以写成\textbf{不带变量}的形式:
\begin{equation}
\label{eqn-with-combinators}
\heartsuit \wedge \mathsf{C} \heartsuit \rightarrow \smiley \mathsf{K}
\end{equation}

但如果 given 的是以下的两个命题:
\begin{eqnarray}
\heartsuit \mbox{ a b} \label{example-1} \\
\heartsuit \mbox{ b a} \label{example-2}
\end{eqnarray}
我不知如何在 algorithm 里应用 (\ref{eqn-with-combinators}) 式。 似乎需要将 (\ref{example-1}) 式 和 (\ref{eqn-with-combinators}) 式 做 pattern matching / unification。  Unification 的结果是将 (\ref{eqn-with-combinators}) 式从右边乘以 \mbox{a b},然后发现它的第二个 conjunct 可以和 (\ref{example-2}) 式  unify。 

据说 combinators 的最大优点就是它消除了 variable binding 的麻烦,但在这例子中,几乎看不到它的优点。  似乎主要的麻烦是 variable substitution,即使没有 bound variables 仍然是麻烦....?

或者这个例子有更好的处理方法(使用 combinators 的)?

\end{document}